\documentclass{article}

\usepackage{listings}
\usepackage{url}
\usepackage{hyperref}
\usepackage{graphicx}
\usepackage{xepersian}
\usepackage{ulem}


\settextfont{XB Zar}
\setlatintextfont{XB Zar}


\makeatletter
\let\@@scshape=\scshape
\renewcommand{\scshape}{%
  \ifnum\strcmp{\f@series}{bx}=\z@
    \usefont{T1}{cmr}{bx}{sc}%
  \else
    \ifnum\strcmp{\f@shape}{it}=\z@
      \fontshape{scsl}\selectfont
    \else
      \@@scshape
    \fi
  \fi}
\makeatother

\title{
مستند فاز سوم پروژه
\\
\vspace{4mm}
سیستم‌های عامل
\\
\vspace{2mm}
دکتر جلیلی
}

\author{
محمدحسین اعلمی
\hspace{1cm}
۹۴۱۰۴۴۰۱
\\
محمدمهدی فاریابی
\hspace{1cm}
۹۳۱۰۱۹۵۱
}

\date{}
\begin{document}

\maketitle

\section*{مقدمه}

این مستند گزارش انجام فاز سوم پروژه درس سیستم‌های عامل به منظور آشنایی با زمان‌بندی پردازه‌ها، مدیریت حافظه و پیاده‌سازی یک شل نمونه است. تمامی عملیات‌های ذکر‌شده در مستند روی سیستم‌عامل
\lr{Ubuntu 16.04}
پیاده و تست‌ شده‌اند.

\section*{گام اول - زمان‌بندی پردازه}



\subsection*{برش‌های زمانی}

در بخش اول گام اول این فاز اطلاعات مربوط به برش‌های زمانی زمان‌بند هسته را از فایل‌های هسته استخراج می‌کنیم.

\begin{thebibliography}{9}

\latin
\bibitem{1}
\url{https://www.qemu.org/}
\end{thebibliography}

\end{document}